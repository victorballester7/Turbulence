\documentclass[10pt,a4paper]{article}
\usepackage[utf8]{inputenc}
\usepackage[left=2cm,right=2cm,top=2cm,bottom=2cm]{geometry}
\usepackage[colorlinks,linkcolor=blue,citecolor=blue,urlcolor=blue]{hyperref}
\usepackage{titlesec}
\usepackage{enumitem}
\usepackage{fancyhdr}
\usepackage{subcaption}
\usepackage{../preamble_math}

\newtheorem{exercici}{Exercice}
\theoremstyle{definition}
\newtheorem{definition}{Definició}
\theoremstyle{remark}
\newtheorem*{res}{Resolution}

\titleformat{\section}
  {\normalfont\fontsize{11}{15}\bfseries}{\thesection}{1em}{}

% \renewcommand{\theenumi}{\textbf{\arabic{enumi}}}
\renewcommand{\theenumi}{\alph{enumi}}
\renewcommand{\theenumiii}{\roman{enumiii}}
\setlength\multlinegap{0pt} % disable the margins on \begin{multline} command.

\title{\bfseries\Large Tutorial 2}

\author{Víctor Ballester Ribó}
\date{\parbox{\linewidth}{\centering
  Turbulence\endgraf
  M2 Applied and Theoretical Mathematics\endgraf
  Université Paris-Dauphine\endgraf
  Februrary 2024\endgraf}}
  %\setlength{\headheight}{13.6pt}

\setlength{\parindent}{0pt}
\begin{document}
\maketitle
\begin{exercici}\label{ex:1}
  Consider the hyper-viscous Navier-Stokes equation
  \begin{equation}
    \partial_t \vf{u} + \vf{u} \cdot \nabla \vf{u} = -\nabla P + (-1)^m \nu_m \nabla^{2m+2} \vf{u} + \vf{F}
  \end{equation}
  where $\nabla^{2m} = \nabla^2 \nabla^2 \overset{m}{\cdots} \nabla^2$.
  \begin{enumerate}
    \item Write an expression for the energy dissipation and the energy balance relation and show that hyper-viscosity dissipates energy (i.e.\ leads to negative term).
    \item Predict the lengthscale $\ell_\nu$ that dissipation becomes effective and dissipates the energy based on the energy injection rate and $\nu_m$.
  \end{enumerate}
\end{exercici}
\begin{res}\hfill
  \begin{enumerate}
    \item We have to compute $\partial_t \mathcal{E}= \partial_t \langle \frac{1}{2} |\vf{u}|^2 \rangle= \langle \vf{u} \cdot \partial_t \vf{u} \rangle$. We have:
          \begin{align*}
            \langle \vf{u} \cdot \partial_t \vf{u} \rangle & = - \langle \vf{u} \cdot (\vf{u} \cdot \nabla \vf{u})\rangle - \langle \vf{u} \cdot \nabla P \rangle + (-1)^m \nu_m \langle \vf{u} \cdot \nabla^{2m+2} \vf{u} \rangle + \langle \vf{u} \cdot \vf{F} \rangle                            \\
                                                           & = -  \left\langle \vf{u} \cdot \nabla \left(\frac{1}{2}|\vf{u}|^2\right) \right\rangle - \langle \vf{u} \cdot \nabla P \rangle  + (-1)^m \nu_m \langle \vf{u} \cdot \nabla^{2m+2} \vf{u} \rangle + \langle \vf{u} \cdot \vf{F} \rangle \\
                                                           & = (-1)^m \nu_m \langle \vf{u} \cdot \nabla^{2m+2} \vf{u} \rangle + \langle \vf{u} \cdot \vf{F} \rangle                                                                                                                                 \\
          \end{align*}
          Now note that $\nabla^{2m+2} \vf{u}$ can be expressed in Einstein notation as $c_{a_1,a_2,a_3}\partial_x^{2a_1} \partial_y^{2a_2} \partial_z^{2a_3} u_i$ for $i=1,2,3$, $a_1+a_2+a_3 = m+1$ and $c_{a_1,a_2,a_3}\geq 0$ are constants.
          Then we have:
          \begin{align*}
            \langle \vf{u} \cdot \nabla^{2m+2} \vf{u} \rangle & = \langle c_{a_1,a_2,a_3}u_i \partial_x^{2a_1} \partial_y^{2a_2} \partial_z^{2a_3} u_i \rangle                                                \\
                                                              & = {(-1)}^{a_1} \langle c_{a_1,a_2,a_3}\partial_x^{a_1} u_i\partial_x^{a_1} \partial_y^{2a_2} \partial_z^{2a_3} u_i \rangle                    \\
                                                              & ={(-1)}^{a_1+a_2} \langle c_{a_1,a_2,a_3}\partial_x^{a_1} \partial_y^{a_2} u_i\partial_x^{a_1} \partial_y^{a_2} \partial_z^{2a_3} u_i \rangle \\
                                                              & ={(-1)}^{a_1+a_2+a_3} \langle c_{a_1,a_2,a_3}{(\partial_x^{a_1} \partial_y^{a_2}\partial_z^{a_3} u_i)}^2 \rangle
          \end{align*}
          where we used that the periodic boundary conditions. Thus:
          $$
            \partial_t\mathcal{E}= (-1)^{-m}{(-1)}^{a_1+a_2+a_3}\nu_m \langle c_{a_1,a_2,a_3}{(\partial_x^{a_1} \partial_y^{a_2} \partial_z^{a_3} u_i)}^2 \rangle + \langle \vf{u} \cdot \vf{F} \rangle= -\nu_m \langle c_{a_1,a_2,a_3}{(\partial_x^{a_1} \partial_y^{a_2} \partial_z^{a_3} u_i)}^2 \rangle + \langle \vf{u} \cdot \vf{F} \rangle
          $$
          whose first term is negative (because we assume $\nu_m\geq 0$), and thus the hyper-viscosity dissipates energy.
    \item Let $\mathcal{I}= \langle \vf{u} \cdot \vf{F} \rangle$ be the energy injection rate. The scaling for the dissipative term is:
          $$
            \nu_m \langle c_{a_1,a_2,a_3}{(\partial_x^{a_1} \partial_y^{a_2} \partial_z^{a_3} u_i)}^2 \rangle \sim \nu_m \frac{1}{\ell^{2(a_1+a_2+a_3)}} {u_{\ell}}^2 \sim \nu_m \frac{1}{\ell^{2m+2}} {u_{\ell}}^2
          $$
          Since $\partial_t \mathcal{E} \sim \frac{{u_\ell}^2}{\tau_\ell}=\frac{{u_\ell}^3}{\ell}$, and on the other hand $\partial_t \mathcal{E} \sim \mathcal{I}$, we have $u_\ell \sim {( \ell \mathcal{I})}^{1/3}$ and so:
          $$
            \nu_m \frac{{u_{\ell}}^2}{\ell^{2m+2}} \sim \mathcal{I} \implies
            \nu_m \frac{{( \ell \mathcal{I})}^{2/3}}{\ell^{2m+2}}\sim \mathcal{I} \implies
            \ell_\nu \sim \left(\frac{\nu_m}{\mathcal{I}^{1/3}}\right)^{1/(2m+4/3)}
          $$
  \end{enumerate}
\end{res}
\begin{exercici}
  Consider again the equation
  \begin{gather}
    \partial_t \vf{b}+\vf{v} \times \vf{b} = -\nabla P + (-1)^m \nu_m \nabla^{2m+2} \vf{b} + \vf{F} \\
  \end{gather}
  in a periodic box of size $L$ where $m$ is an integer and $\vf{F}$ is a forcing that injects energy at scale $L$ at a rate $\mathcal{I}$. $\vf{v}$ is related to $\vf{b}$ as $\vf{v} = {(\rotp)}^n \vf{b}$ for some $n \in \NN$. Recall that for any $n$, the energy $\mathcal{E}= \langle \frac{1}{2} |\vf{b}|^2 \rangle$ is conserved for $\nu_m = 0$ and $\alpha = 0$. Assuming:
  \begin{itemize}
    \item Energy cascades to smaller scales
    \item Similar size eddies dominate the cascade
  \end{itemize}
  show the following:
  \begin{enumerate}
    \item Predict the energy spectrum of $\vf{b}$ based on the assumptions above.
    \item Predict the lengthscale $\ell_\nu$ that dissipation becomes effective and dissipates the energy.
    \item For which values of $n$ and $m$, the viscosity will not be sufficient to dissipate the injected energy as $\nu_m \to 0$?
  \end{enumerate}
\end{exercici}
\begin{res}\hfill
  \begin{enumerate}
    \item As in exercise \ref{ex:1} we have that:
          $$
            \partial_t \mathcal{E} = -\nu_m \langle c_{a_1,a_2,a_3}{(\partial_x^{a_1} \partial_y^{a_2} \partial_z^{a_3} b_i)}^2 \rangle + \langle \vf{b} \cdot \vf{F} \rangle
          $$
          Since the units of each term in the equation are the same, we have the following scalings:
          $$
            \frac{b}{\tau} \sim \frac{b^2}{\ell^n} \implies \frac{1}{\tau} \sim \frac{b}{\ell^n}
          $$
          Now, we have that $\epsilon\sim \Pi_{\vf{b}}$, where $\Pi_{\vf{b}}$ is the flux of energy of $\vf{b}$. Thus:
          $$
            \epsilon\sim \Pi_{\vf{b}}\sim \frac{b_\ell^2}{\tau_\ell} \sim \frac{b_\ell^3}{\ell^n} \sim \implies b_\ell \sim \left(\ell^{n}\epsilon\right)^{1/3}
          $$
          Thus, the energy spectrum of $\vf{b}$ is:
          $$
            E(k) \sim \frac{b_\ell^2}{1/\ell} \sim \ell^{2n/3+1} \epsilon^{2/3}=\epsilon^{2/3} k^{-(3+2n)/3}
          $$
    \item When viscosity becomes effective, we have (using exercise \ref{ex:1}):
          $$
            \epsilon \sim \nu_m \frac{{b_{\ell_\nu}}^2}{\ell_\nu^{2m+2}}  \sim \nu_m \frac{\ell_\nu^{2n/3} \epsilon^{2/3}}{\ell_\nu^{2m+2}}  \sim \nu_m \ell_\nu^{2n/3-2m-2} \epsilon^{2/3}\implies \ell_\nu \sim \left(\frac{\nu_m}{\epsilon^{1/3}}\right)^{1/(2+2m-2n/3)}
          $$
    \item If $\nu_m \to 0$, then we must have $\ell_\nu \to \infty$ if we do not want the viscosity to dissipate the energy. This implies that the exponent of $\ell_\nu$ must be negative, i.e.\ $2+2m-2n/3<0$, which is equivalent to $n>3(m+1)$.
  \end{enumerate}
\end{res}
\end{document}


